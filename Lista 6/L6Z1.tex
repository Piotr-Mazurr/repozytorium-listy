%nie chce wyświetlać polskich znaków w tabelach
\documentclass{article}

% użyte pakiety
\usepackage{polski}
\usepackage[utf8]{inputenc}
\usepackage{amsmath}
\usepackage{amssymb}
\usepackage{hyperref}
\usepackage[left=3cm, right=3cm, top=2cm, bottom=3cm]{geometry}

\usepackage{listings}
\lstset{frame=single}

\newtheorem{theorem}{Twierdzenie}[section]
\newtheorem{definition}[theorem]{Definicja}

\begin{document}
    
\section{Listy}
\textbf{Przykład 1: nieuporządkowana lista (itimize)}

\noindent \textbf{Kod:}
\begin{lstlisting}
\begin{lstlisting}
    \item Akacja
    \item Cis
\end{itemize}
\end{lstlisting}

\noindent \textbf{Wyjście:}
\begin{itemize}
    \item Akacja
    \item Baobab
    \item Cis
\end{itemize}

\noindent Natomiast używając \textbf{enumerate} zamiast \textbf{itemize:}
\begin{enumerate}
    \item Akacja
    \item Baobab
    \item Cis
\end{enumerate}

\bigskip

\section{Wyrażenia matematyczne}
\textbf{Przykład 2: wyrażenia "w linii" i we własnym paragrafie}

\noindent \textbf{Kod}:
\begin{lstlisting}
W linii: \(a^2 + b^2 = c^2.\)

Wycentrowane:
\[
\sum_{i=1}^\infty \frac{1}{n} = \infty
\]
\end{lstlisting}

\noindent \textbf{Wyjście:}

W linii: \(a^2 + b^2 = c^2 .\)

Wycentrowane :
\[
\sum_{i=1}^\infty \frac{1}{n} = \infty.
\]

\noindent \textbf{inne przykłady:}

\hspace{4cm} \(\sin^{2}\left(\alpha\right)+\cos^{2}\left(\alpha\right)=1.\)
\[
x\in A \cap B \iff x \in A \wedge x \in B. \\
\]
\[
\emptyset \times A = \emptyset. \\
\]

\[
\forall x \in \mathbb{R} \ \exists n\in \mathbb{N}_{>0} \ \frac{1}{n}<|x|.
\]

\section{Środowiska}
\textbf{Przykład 3: Definicje i twierdzenia}

\noindent Sprawdź w preambule \verb|wyklad6_latex.tex| w jaki sposób zostały użyte makra \verb|\newtheorem.|

\noindent Następnie

\noindent \textbf{Kod:}

\begin{lstlisting}
\begin{definition}
Liczba \(n \in \mathbb{N}\) jest pierwsza, gdy \(n>1\) i nie
istnieje \(1<d<n\) takie, że \(d\) dzieli \(n\
\end{definition}

\begin{theorem}
Istnieje nieskońćzenie wiele liczb pierwszych
\end{theorem}
\end{lstlisting}

\noindent \textbf{Wyjście:}

\begin{definition}
Liczba \(n \in \mathbb{N}\) jest pierwsza, gdy \(n>1\) i nie istnieje \(1<d<n\) takie, że \(d\) dzieli \(n\)
\end{definition}

\begin{theorem}
Istnieje nieskońćzenie wiele liczb pierwszych
\end{theorem}

\noindent(Zwróć uwagę na sposób numeracji - żeby wyszła tak jak powyższa, musi być oparta o \textbf{section} zamiast \textbf{subsection})

\section{Fragmenty kodu}
w poprzednich przykładach pojawiają się obramowane fragmenty kodu: dla nich też istnieje specjalne środowisko. Umieść w preambule fragment:
\begin{lstlisting}
\usepackage{listings}
\lstset{frame=single}
print(i)
\end{lstlisting}

\noindent Następnie kod można umieszczać w środowisku \textbf{lstlisting:}

\noindent \textbf{Przykład 4: blok kodu}

\noindent \textbf{Kod:}

\begin{lstlisting}
\begin{lstlisting}
for i in range (10):
print(i)
\textbackslash end{lstlisting}
\end{lstlisting}

\noindent \textbf{Wyjście:}

\begin{lstlisting}
for i in range (10):
    print(i)
\end{lstlisting}

\noindent Kod można też umieszczać "w linii" używając \verb|\verb|, np.

\noindent \textbf{Kod:}

\begin{lstlisting}
ten fragment mówi coś o makrze i podaje 
przykład użycia.
\end{lstlisting}


\noindent \textbf{Wyjście}: ten fragment mówi coś o makrze \verb|\lstlisting| i podaje przykład użycia.

\end{document}


